\chapter{Tổng quan}

\section{Nhu cầu quản lý tài nguyên}

Nhu cầu quản lý tài nguyên tài liệu hướng dẫn là nhu cầu có thực và phổ biến hiện nay. Hiện nay việc tổ chức các kho tài nguyên như thế đã dần hoàn thiện và số lượng được lưu trữ của các tài liệu ngày càng được gia tăng.
Với một số lượng khổng lồ các kho tài nguyên, việc tìm kiếm chính xác không phải là việc đơn giản. Việc tìm kiếm theo hướng ngữ nghĩa vẫn còn hạn chế. 
Việc tra cứu trước đây đến nay đa số dựa trên các siêu dữ liệu liên quan trong trong danh mục tài liệu mà ít có ứng dụng ngữ nghĩa vào việc tìm kiếm. Việc tìm kiếm không chính xác gây khó khăn cho người sử dụng và hiệu quả không cao, khai thác không được triệt để nguồn tài liệu thông tin.

Một vấn đề khác với các hệ thống tìm kiếm sách là ứng dụng một mô hình với mọi loại tài liệu thuộc nhiều lĩnh vực khác nhau. Trong khi có người tìm kiếm thường chỉ quan tâm đến một số thể loại nhất định tùy trường hợp. Việc sử dụng chung mô hình khó chính xác cao do đặc thù riêng, giảm hiệu quả đối với nhu cầu tìm kiếm thực tế.


\section{Mục tiêu báo cáo}

Xuất phát từ yêu cầu trên, báo cáo nhằm mục tiêu tìm hiểu xây dựng ứng dụng tìm kiếm cho hệ thống quản lý ebook trực tuyến. Đồng thời báo cáo tập trung vào một số thể loại sách cụ thể. 

Công việc cụ thể:
\begin{itemize}
    \item Có một hệ thống ebook online có sẵn, rút ra đặc thù riêng cho từng thể loại.
    \item Nghiên cứu phương pháp biểu diễn ngữ nghĩa của tài liệu để áp dụng vào quản lý kho ebook, mô hình Ontology mô tả tri thức lĩnh vực.
    \item Công cụ hỗ trợ xây dựng hệ thống tìm kiếm theo ngữ nghĩa.
\end{itemize}

\section{Phạm vi báo cáo}

Báo cáo chủ yếu tập trung tìm hiểu về sử dụng phương pháp Ontology vào xây dựng, tối ưu một hệ thống tìm kiếm có sẵn bằng cách bổ sung ngữ nghĩa, giúp cho việc tìm kiếm chính xác hơn. Lĩnh vực tìm kiếm là trên dữ liệu hệ thống quản lý chia sẻ ebook.


